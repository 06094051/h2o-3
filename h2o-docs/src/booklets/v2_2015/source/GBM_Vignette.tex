% -*- mode: tex; fill-column: 115; -*-

\documentclass[a4paper, twoside]{article}
\setcounter{secnumdepth}{5}
\setcounter{tocdepth}{5}
\usepackage[english]{babel}
\usepackage{textcomp}
\usepackage{amsmath,amsthm,amsfonts,amssymb,epsfig}
\usepackage{array}
\usepackage{datetime}
\usepackage{lipsum}% http://ctan.org/pkg/lipsum
\usepackage[left=1.1in,top=1in,right=1.1in]{geometry}% http://ctan.org/pkg/geometry
\usepackage{listings}% http://ctan.org/pkg/listings
\usepackage{spverbatim}
\usepackage{hyperref}
\usepackage{microtype}
\hypersetup{colorlinks=true, urlcolor=black, linkcolor=black}
\usepackage{graphicx}
\graphicspath{ {images/} }
\usepackage{parskip}
\usepackage{titlesec} %used for diminishing heading sizes
\usepackage[square, sort, comma, numbers]{natbib} %%uses titles for cited references
\usepackage[fit]{truncate}
\usepackage{fancyhdr}
\pagestyle{fancy}
\fancyhead{}
\fancyhead[RO, RE]{\thepage}
\fancyhead[LO, LE]{\rightmark}
\renewcommand{\sectionmark}[1]{\markboth{}{\textsc{\thesection~#1}}}
\fancyfoot[C]{}%hide footer
\usepackage{xcolor} 
\usepackage[scaled=1]{couriers}

\xdefinecolor{gray}{rgb}{0.6,0.6,0.6} 


\titleformat*{\section}{\LARGE\bfseries\sffamily}
\titleformat*{\subsection}{\Large\bfseries\sffamily}
\titleformat*{\subsubsection}{\large\bfseries\sffamily}
\titleformat*{\paragraph}{\large\bfseries\sffamily}
\titleformat*{\subparagraph}{\large\bfseries\sffamily}
\renewcommand{\familydefault}{\sfdefault} %sans-serif font



\begin{document}


%----------------------------------------------------------------------
% Definition for "lstlisting" blocks
%----------------------------------------------------------------------
% --- USAGE ---
%
% \begin{lstlisting}[style=R}
% ...
% \end{lstlisting}
%
% % \begin{lstlisting}[style=output}
% ...
% \end{lstlisting}
%----------------------------------------------------------------------

% By default, make listings all black so it's easy to spot the ones that aren't set to a style.
% This is just a debugging technique.
\lstset{backgroundcolor=\color{black}}

% Define scala language first
% ``define'' Scala
\lstdefinelanguage{scala}{
  morekeywords={abstract,case,catch,class,def,%
    do,else,extends,false,final,finally,%
    for,if,implicit,import,match,mixin,%
    new,null,object,override,package,%
    private,protected,requires,return,sealed,%
    super,this,throw,trait,true,try,%
    type,val,var,while,with,yield},
  otherkeywords={=>,<-,<\%,<:,>:,\#,@},
  sensitive=true,
  morecomment=[l]{//},
  morecomment=[n]{/*}{*/},
  morestring=[b]``,
  morestring=[b]',
  morestring=[b]''``
}

\lstdefinestyle{R}{
  language=R,
  frame=single,
  breaklines,
  basicstyle=\ttfamily,
  commentstyle=\textbf,% comment style
  keywordstyle=\ttfamily,
  numbers=left,% display line numbers on the left side 
  numberstyle=\scriptsize,% use small line numbers 
  numbersep=10pt,% space between line numbers and code
  backgroundcolor=\color{white}, 
  showstringspaces=false % don't show spaces as weird char.
}

\lstdefinestyle{python}{
  language=python,
  frame=single,
  breaklines,
  basicstyle=\ttfamily,
  commentstyle=\textsl,% comment style
  keywordstyle=\ttfamily,
  numbers=left,% display line numbers on the left side 
  numberstyle=\scriptsize,% use small line numbers 
  numbersep=10pt,% space between line numbers and code
  backgroundcolor=\color{white}, 
  showstringspaces=false %don't show spaces as weird char.
}

\lstdefinestyle{Scala}{
  language=scala,
  frame=single,
  breaklines,
  basicstyle=\ttfamily,
  commentstyle=\textsl,% comment style
  keywordstyle=\ttfamily,
  numbers=left,% display line numbers on the left side 
  numberstyle=\scriptsize,% use small line numbers 
  numbersep=10pt,% space between line numbers and code
  backgroundcolor=\color{white}, 
  showstringspaces=false % don't show spaces as weird char.
}

\lstdefinestyle{Bash}{
  language=bash,
  frame=single,
  breaklines,
  basicstyle=\ttfamily,
  commentstyle=\textsl,% comment style
  keywordstyle=\ttfamily,
  numbers=left,% display line numbers on the left side 
  numberstyle=\scriptsize,% use small line numbers 
  numbersep=10pt,% space between line numbers and code
  backgroundcolor=\color{white}, 
  showstringspaces=false % don't show spaces as weird char.
}


\definecolor{mygray}{rgb}{0.92,0.92,0.92}

\lstdefinestyle{output}{
  frame=single,
  breaklines,
  basicstyle=\ttfamily,
  numbers=left,% display line numbers on the left side 
  numberstyle=\scriptsize,% use small line numbers 
  numbersep=10pt,% space between line numbers and code
  backgroundcolor=\color{mygray}, 
  showstringspaces=false %don't show spaces as weird char.
}

\newcommand{\waterExampleInR} {
\textbf{Example in R} \\
}

\newcommand{\waterExampleInPython} {
\textbf{Example in Python} \\
}


\thispagestyle{empty} %removes page number

\begin{center}
\textsc{\Large\bf{Gradient Boosted Machines with H2O}}
\\
\bigskip
\textsc{\small{Cliff Click \hspace{20pt} Michal Malohlava \hspace{20pt} Viraj Parmar \hspace{20pt} Hank Roark \hspace{20pt} Jessica Lanford}}
\\
\bigskip
\line(1,0){250}  %inserts  horizontal line

{\url{http://h2o.gitbooks.io/gbm-with-h2o/}}

\bigskip
August 2015: Third Edition 
\\%add front page image here? (wavy lines)
\bigskip
\end{center}

{\raggedright\vfill\ 

Gradient Boosted Machines with H2O\\
  by Cliff Click, Michal Malohlava, Viraj Parmar, Hank Roark, \&\ Jessica Lanford \\
\bigskip
  Published by H2O.ai, Inc. \\
2307 Leghorn St. \\
Mountain View, CA 94043\\
\bigskip
\textcopyright 2015 H2O.ai, Inc. All Rights Reserved. 
\bigskip

August 2015: Third Edition
\bigskip

Photos by \textcopyright H2O.ai, Inc.
\bigskip

While every precaution has been taken in the\\
preparation of this book, the publisher and\\
authors assume no responsibility for errors or\\
omissions, or for damages resulting from the\\
use of the information contained herein.\\
\bigskip
Printed in the United States of America. 
}

\newpage

\tableofcontents

%----------------------------------------------------------------------
%----------------------------------------------------------------------

\newpage

\section{Introduction}
This document introduces the reader to Gradient Boosted Machines (GBM) with H2O.  
Examples are written in R and Python.
The reader is walked through the installation of H2O, basic GBM concepts, building GBM models in H2O, how to
interpret model output, how to make predictions, and various implementation details.

%----------------------------------------------------------------------
%----------------------------------------------------------------------

\section{What is H2O?}
\Urlmuskip=0mu plus 1mu\relax %needed to make long URLs break nicely


H2O is fast, scalable, open-source machine learning and deep learning for smarter applications. With H2O, enterprises like PayPal, Nielsen Catalina, Cisco, and others can use all their data without sampling to get accurate predictions faster. Advanced algorithms such as deep learning, boosting, and bagging ensembles are built-in to help application designers create smarter applications through elegant APIs. Some of our initial customers have built powerful domain-specific predictive engines for recommendations, customer churn, propensity to buy, dynamic pricing, and fraud detection for the insurance, healthcare, telecommunications, ad tech, retail, and payment systems industries.

Using in-memory compression, H2O handles billions of data rows in-memory, even with a small cluster. To make it easier for non-engineers to create complete analytic workflows, H2O's platform includes interfaces for R, Python, Scala, Java, JSON, and CoffeeScript/JavaScript, as well as a built-in  web interface, Flow. H2O was built alongside (and on top of) Hadoop and Spark Clusters and typically deploys within minutes.

H2O includes many common machine learning algorithms, such as generalized linear modeling (linear regression, logistic regression, etc.), Na\"{i}ve Bayes, principal components analysis, time series, k-means clustering, and others. H2O also implements best-in-class algorithms at scale, such as distributed random forest, gradient boosting and deep learning. Customers can build thousands of models and compare the results to get the best predictions.

H2O is nurturing a grassroots movement of physicists, mathematicians, and computer scientists to herald the new wave of discovery with data science by collaborating closely with academic researchers and Industrial data scientists. Stanford university giants Stephen Boyd, Trevor Hastie, Rob Tibshirani advise the H2O team on building scalable machine learning algorithms. With hundreds of meetups over the past three years, H2O has become a word-of-mouth phenomenon, growing amongst the data community by a hundred-fold, and is now used by 30,000+ users and is deployed using R, Python, Hadoop, and Spark in 2000+ corporations.

\textbf{Try it out}

\begin{itemize}
\item  Download H2O directly at \mbox{\url{http://h2o.ai/download}}.
\item Install H2O's R package from CRAN at {\url{https://cran.r-project.org/web/packages/h2o/}}. 
\item Install the Python package from PyPI at {\url{https://pypi.python.org/pypi/h2o/}}.

\end{itemize}



\textbf{Join the community}
\begin{itemize}
\item  To learn about our meetups, training sessions, hackathons, and product updates, visit {\url{http://h2o.ai}}. 
\item Visit the open source community forum at {\url{https://groups.google.com/d/forum/h2ostream}}.
\item Join the chat at {\url{https://gitter.im/h2oai/h2o-3}}.
\end{itemize}





%----------------------------------------------------------------------
%----------------------------------------------------------------------

\newpage

\newcommand{\waterVersion}{3.0.1.4}

\section{Installation} 
\Urlmuskip=0mu plus 1mu\relax %needed to make long URLs break nicely


The easiest way to directly install H2O is  via an R or Python package.

({\bf{Note}}: The examples in this document were created with H2O version \waterVersion.)

\subsection{Installation in R}

To load a recent H2O package from CRAN, run:

\begin{lstlisting}[style=R]
install.packages("h2o")
\end{lstlisting}

{\bf{Note}}: The version of H2O in CRAN is often one release behind the current version.

For the latest recommended version, download the
latest stable H2O-3 build from the H2O download page:

\begin{enumerate}
\item Go to {\url{http://h2o.ai/download}}.
\item Choose the latest stable H2O-3 build.
\item Click the ``Install in R'' tab.
\item Copy and paste the commands into your R session.
\end{enumerate}

\bigskip
After H2O is installed on your system, verify the installation:

\begin{lstlisting}[style=R]
library(h2o)

#Start H2O on your local machine using all available cores.
#By default, CRAN policies limit use to only 2 cores.
h2o.init(nthreads = -1)

#Get help
?h2o.glm
?h2o.gbm

#Show a demo
demo(h2o.glm)
demo(h2o.gbm)
\end{lstlisting}

\subsection{Installation in Python}

To load a recent H2O package from PyPI, run:

\begin{lstlisting}[style=python]
pip install h2o
\end{lstlisting}

To download the
latest stable H2O-3 build from the H2O download page:

\begin{enumerate}
\item Go to {\url{http://h2o.ai/download}}.
\item Choose the latest stable H2O-3 build.
\item Click the ``Install in Python'' tab.
\item Copy and paste the commands into your Python session.
\end{enumerate}

\bigskip
After H2O is installed, verify the installation:

\begin{lstlisting}[style=python]
import h2o

# Start H2O on your local machine
h2o.init()

# Get help
help(h2o.glm)
help(h2o.gbm)

# Show a demo
h2o.demo("glm")
h2o.demo("gbm")

\end{lstlisting}

\subsection{Pointing to a Different H2O Cluster}

The instructions in the previous sections create a one-node H2O cluster on your local machine. 

To connect to an established H2O cluster (in a multi-node Hadoop environment, for example) specify the IP address and port number for the established cluster using the \texttt{ip} and \texttt{port} parameters in the \texttt{h2o.init()} command.  The syntax for this function is identical for R and Python:
\medskip  

\begin{lstlisting}[style=R]
h2o.init(ip = "123.45.67.89", port = 54321)
\end{lstlisting}

%if it's the same, only one is needed



\subsection{Example code}

R code for the examples in this document are available here:

\url{https://github.com/h2oai/h2o-3/blob/master/h2o-docs/src/booklets/v2_2015/source/GBM_Vignette.R}

Python code for the examples in this document can be found here:

\url{https://github.com/h2oai/h2o-3/blob/master/h2o-docs/src/booklets/v2_2015/source/GBM_Vignette.py}

The document source itself can be found here:

\url{https://github.com/h2oai/h2o-3/blob/master/h2o-docs/src/booklets/v2_2015/source/GBM_Vignette.tex}

%----------------------------------------------------------------------
%----------------------------------------------------------------------
\section{Gradient Boosting Overview}

A gradient boosted model is an ensemble of tree models that can be either regression or classification.
Both are forward-learning ensemble methods that obtain predictive results through gradually improved estimations.
Boosting is a flexible nonlinear regression procedure that helps improve the accuracy of trees.
By sequentially applying weak classification algorithms to the incrementally changed data,
a series of decision trees are created that produce an ensemble of weak prediction models.
While boosting trees increases their accuracy, it also decreases speed and human interpretability.
The gradient boosting method generalizes tree boosting to minimize these issues.

\subsection{Summary of features}
H2O's GBM functionalities include:

\begin{itemize}
\item supervised learning for regression and classification tasks
\item distributed and parallelized computation on either a single node or a multi-node cluster
\item fast and memory-efficient Java implementations of the underlying algorithms
\item user-friendly web interface to mirror the model building and scoring process running in R or Python
\item grid search for hyperparameter optimization and model selection
\item model export in plain Java code for deployment in production environments
\item additional parameters for model tuning
\end{itemize}

\subsection{ Model Parameters}

This section describes the functions of the parameters for GBM (TODO, lots of params, maybe copy to appendix and only put most important ones here):

\begin{itemize}
\item {\texttt{x}}: A vector containing the names of the predictors to use while building the GBM model. No default.
\item {\texttt{y}}: A character string or index that represents the response variable in the model. No default.
\item {\texttt{training\_frame}}: An \texttt{H2OFrame} object containing the variables in the model. 
\item {\texttt{validation\_frame}}: An \texttt{H2OParsedData} object containing the validation dataset used to construct confusion matrix. If  blank, the training data is used by default.
\item {\texttt{nfolds}}: Number of folds for cross-validation. If \texttt{nfolds >=2}, then \texttt{validation\_frame} must remain blank. Default is 0. %%validation_frame = correct? just says "validation" in R doc...
\item {\texttt{ignore\_const\_cols}}: A boolean indicating if constant columns should be ignored.  Default is True.
\item {\texttt{ntrees}}: A non-negative integer that defines the number of trees. The default is 50.
\item {\texttt{max\_depth}}: The user-defined tree depth. The default is 5.
\item {\texttt{min\_rows}}: The minimum number of rows to assign to the terminal nodes. The default is 10.
\item {\texttt{nbins}}: For numerical columns (real/int), build a histogram of (at least) this many bins, then split at the best point The default is 20.
\item {\texttt{nbins\_cats}}: For categorical columns (enum), build a histogram of this many bins, then split at the best point. Higher values can lead to more overfitting.  The default is 1024
\item {\texttt{seed}}: Seed for random numbers that affect sampling.
\item {\texttt{learn\_rate}}: An integer that defines the learning rate. The default is 0.1 and the range is 0.0 to 1.0.
\item {\texttt{distribution}}: Enter {\texttt{AUTO, bernoulli, multinomial, gaussian, poisson, gamma}} or {\texttt{tweedie}} to select the distribution function. The default is {\texttt{AUTO}}.
\item {\texttt{score\_each\_iteration}}: A boolean indicating whether to score during each iteration of model training.  Default is false.
\item \texttt{fold\_assignment}: Cross-validation fold assignment scheme, if \texttt{fold\_column} is not specified. The following options are supported: \texttt{AUTO, Random,} or \texttt{Modulo}. 
\item \texttt{fold\_column}:  Column with cross-validation fold index assignment per observation. 
\item \texttt{offset\_column}: Specify the offset column. Note: Offsets are per-row “bias values” that are used during model training. For Gaussian distributions, they can be seen as simple corrections to the response (y) column. Instead of learning to predict the response (y-row), the model learns to predict the (row) offset of the response column. For other distributions, the offset corrections are applied in the linearized space before applying the inverse link function to get the actual response values. 
\item \texttt{weights\_column}: Specify the weights column. Note: Weights are per-row observation weights. This is typically the number of times a row is repeated, but non-integer values are supported as well. During training, rows with higher weights matter more, due to the larger loss function pre-factor.
\item {\texttt{balance\_classes}}: Balance training data class counts via over or undersampling for imbalanced data. The default is {\texttt{FALSE}}.
\item {\texttt{max\_confusion\_matrix\_size}}: Maximum size (number classes) for confusion matrices to be printed in the H2O logs.  Default it 20.
\item {\texttt{max\_hit\_ratio\_k}}: Maximum number (top K) of predictions to use for hit ratio computation.  Use for multi-class only, 0 to disable.  Default is 10.
\item {\texttt{r2\_stopping}}: Stop making trees when the $R^2$ metric equals or exceeds this value.  Default is 0.999999.
\item {\texttt{build\_tree\_one\_node}}: Boolean indicating if GBM will be run on one node only; no network overhead but fewer cpus used. Suitable for small datasets.  Default is False.
\item {\texttt{tweedie\_power}}: A numeric specifying the power for the tweedie function when \texttt{distribution = "tweedie"}.  Default is 1.5.
\item {\texttt{checkpoint}}: TODO
\item {\texttt{keep\_cross\_validation\_predictions}}: Specify whether to keep the predictions of the cross-validation models.   Default is False.
\item {\texttt{class\_sampling\_factors}}: Desired over/under-sampling ratios per class (in lexicographic order). If not specified, sampling factors will be automatically computed to obtain class balance during training. Requires \texttt{balance\_classes}.
\item {\texttt{max\_after\_balance\_size}}: Maximum relative size of the training data after balancing class counts; can be less than 1.0.  The default is 5.
\item {\texttt{nbins\_top\_level}}: For numerical columns (real/int), build a histogram of (at most) this many bins at the root level, then decrease by factor of two per level.
\item \texttt{model\_id}: The unique ID assigned to the generated model. If not specified, an ID is generated automatically.
\end{itemize}



\subsection{Theory and framework}

Gradient boosting is a machine learning technique that combines two powerful tools: gradient-based optimization and
boosting. Gradient-based optimization uses gradient computations to minimize a model's loss function with respect to
the training data. Boosting additively collects an ensemble of weak models in order to ultimately create a strong
learning system for predictive tasks. Here we consider gradient boosting in the example of K-class classification,
although the model for regression follows a similar logic. The following analysis follows from the discussion in
Hastie et al (2010) at {\url{http://statweb.stanford.edu/~tibs/ElemStatLearn/}}.

\newline
\\
{\bf{\footnotesize{GBM for classification}}}
\\
\line(1,0){500}
\\
1. Initialize $f_{k0} = 0, k = 1,2,\dots,K$
\\
2. For $m=1$ to $M$

\hspace{1cm} a. Set $p_k(x) = \frac{e^{f_k(x)}}{\sum_{l=1}^K e^{f_l(x)}}$ for all $k = 1,2\dots, K$

\hspace{1cm} b. For $k=1$ to $K$

\hspace{2cm} i. Compute $r_{ikm} = y_{ik} - p_k(x_i),  i = 1,2,\dots,N$

\hspace{2cm} ii. Fit a regression tree to the targets $r_{ikm}, i = 1,2,\dots,N$,
\par \hspace{2.5cm} giving terminal regions $R_{jkm}, 1,2,\dots,J_m$

\hspace{2cm}iii. Compute $$\gamma_{jkm} = \frac{K-1}{K} \frac{\sum_{x_i \in R_{jkm}} (r_{ikm})}{\sum_{x_i \in R_{jkm}} |r_{ikm}| (1 - |r_{ikm}|)} , j=1,2,\dots,J_m$$

\hspace{2cm} iv. Update $f_{km}(x) = f_{k,m-1}(x) + \sum_{j=1}^{J_m} \gamma_{jkm} I(x \in R_{jkm})$
\\
3. Output $f_k^{\hat{}}(x) = f_{kM}(x),  k=1,2,\dots,K$
\\
\line(1,0){500}

In the above algorithm, for $k$-classification, H2O builds $k$-regression trees to represent one classification tree. The index $m$ tracks of the number of weak learners added to the current ensemble. Within this outer loop, there is an inner loop across each of the $K$ classes. In this inner loop, the first step is to compute the residuals, $r_{ikm}$, which are actually the gradient values, for each of the $N$ bins in the CART model, and then to fit a regression tree to these gradient computations. This fitting process is distributed and parallelized, and details on this framework can be found on the h2o.ai blog at {\url{http://h2o.ai/blog/2013/10/building-distributed-gbm-h2o/}}.
\\
\\
The final procedure in the inner loop is to add to the current model to the fitted regression tree, which improves the accuracy of the model during the inherent gradient descent step. After $M$ iterations, the final ``boosted" model can be tested out on new data.

\subsection{Loss Function}
The AdaBoost method builds an additive logistic regression model:
$${F(x) = log}\frac{Pr(Y = 1|x)}{Pr(Y = -1|x)} = \sum_{m=1}^{M} \alpha_m f_m (x) $$

by stagewise fitting using the loss function:
$$L(y, F(x)) = exp(-y  F (x)) $$


%---- TODO: Check that this is the same
\subsection{Distributed Trees}

H2O's implementation of GBM uses distributed trees. H2O overlays trees on the data by assigning a tree node to each row.
The nodes are numbered and the number of each node is stored in a temporary vector as "Node\_ID" for each row. H2O makes
a pass over all the rows using the most efficient method, which may not necessarily be  numerical order. A local
histogram using only local data is created in parallel for each row on each node. The histograms are then assembled and a split column is selected to make the decision. The rows are re-assigned to nodes and the entire process is repeated.

For example, for an initial tree, all rows start on node 0. A MapReduce (MR) task computes the statistics and uses
them to make an algorithmically-based decision, such as lowest mean squared error (MSE). In the next layer in the
tree (and the next MR task), a decision is made for each row: if X $<$ 1.5, go right in the tree; otherwise, go left.
H2O computes the stats for each new leaf in the tree, and each pass across all the rows builds the entire layer.

For multinomial or binomial, the split is determined by the number of columns. The number of columns is evaluated to
find the best split out of the possible combinations. For example, for a hundred-column dataset that uses twenty bins,
there are 2000 (20x100) possible split points.

Each layer represents another MR task; therefore, a tree that is five layers deep requires five passes. Each tree
level is fully data-parallelized. Each pass is over one layer in the tree, and builds a per-node histogram in the
MR calls. As each pass analyzes a tree level, H2O then decides how to build the next level. H2O reassigns rows to
new levels in another pass by merging the two passes and builds a histogram for each node. Each per-level histogram
is done in parallel.

Scoring and building is done in one pass. Each row is tested against the decision from the previous pass, assigned
to a new leaf, and a histogram is built on that leaf. To score, H2O traverses the tree and obtains the results. The
tree is compressed to a smaller object that can still be traversed, scored, and printed.

Although the GBM algorithm builds each tree one level at a time, H2O is able to quickly run the entire level in
parallel and distributed. The processing requirements for more data can be offset by more CPUs or nodes.
Since H2O does the per-level compute in parallel, which requires sending histograms over the network, the amount
of data can become very large for a very deep tree.

High max-bin sizes slow down generation of deep trees, but this is not a problem for GBM tree depths as GBM
typically uses very shallow trees. (TODO: Check, this probably isn't true anymore)  Bin limits over 20 have been shown to provide very little improvement on
the quality of the GBM model.

For the MSE reports, the zero-tree report uses the class distribution as the prediction. The one-tree report
uses the first tree, so the first two reports are not equal. The reported MSE is the inclusive effect of all
prior trees, and generally decreases monotonically on the training dataset. However, the curve will generally
bottom out and then begin to slowly rise on the validation set as overfitting sets in.

The computing cost is based on the number of leaves, but depending on the dataset, the number of leaves can be
difficult to predict. The maximum number of leaves is $2^d$, where $d$ represents the tree depth.

%---- TODO: Check that this is the same
\subsection{Treatment of factors}

When the specified GBM model includes factors, those factors are analyzed by assigning an integer to each distinct
factor level, and then binning the ordered integers according to the user-specified number of bins ($N$ bins). Split
points are determined by considering the end points of each bin and the one-versus-many split for each bin.

For example, if the factor is split into five bins, H2O orders the bins by bin number, then considers the split b
etween the first and second bin, then the second and third, then the third and fourth, and the fourth and fifth.
Additionally, the split that results from splitting the first bin from the other four and all analogous splits for
the other four bins are considered. To specify a model that considers all factors individually, set the value for
$N$ bins equal to the number of factor levels. This can be done for over 1024 levels (the maximum number of levels
that can be handled in R), though this increases the time to fully generate a model.

Increasing the number of bins is less useful for covering factor columns, but is more important for the
one-versus-many approach. The "split-by-a-numerical-value" is basically a random split of the factors, so the
number of bins is less important. Top-level tree splits (shallow splits) use the maximum allotment as their bin size,
so the top split uses 1024 bins, the next level in the tree uses 512 bins, and so on.

Factors for binary classification have a third (and optimal) choice: to split all bins (and factors within those bins)
that have a mean of less than 0.5 one way, and the rest of the bins and factors the other way, creating an arbitrary
set of factors with a known optimal split. This is represented as a bitset in the embedded Java model. Therefore,
factor columns with less than the limit ($\mathtt{\sim}$ 1024) always get an optimal split for categorical problems.

For categorical problems with $N$ possible values, the split candidate is determined by the formula $2^{N-1}-1$.
For binary classification and regression problems, the number of split candidates is reduced to $N-1$ by sorting
the categorical feature values by label average.

%---- TODO: Check that this is the same
\subsection{Key parameters}

In the above example, an important user-specified value is $N$, which represents the number of bins that data are
partitioned into before the tree's best split point is determined. Split points are determined by considering the
end points of each bin, and the one-versus-many split for each bin. To model all factors individually, you can
specify high $N$ values, but this will slow down the modeling process. For shallow trees, we recommend keeping the
total count of bins across all splits at 1024 (so that a top-level split uses 1024, but a 2nd level split uses 512
bins, and so forth). This value is then maxed with the input bin count.
\\
\\
Another important parameter to specify is the size ($J$) of the trees, which must be controlled in order to avoid
overfitting. Increasing $J$ enables larger variable interaction effects, so knowing about these effects is helpful
in setting the value for $J$. Large values of $J$ have also been found to have excessive computational cost,
since Cost = \#columns $\cdot N \cdot K \cdot 2^{J}$. However, lower values generally also have the highest
performance. Models with $4 \leq J \leq 8$ and a larger number of trees $M$ reflect this generalization.
Later, we will discuss how to use grid search models to tune these parameters in the model selection process.
\\
\\
You can also specify the shrinkage constant, which controls the learning rate of the model and is actually a
form of regularization. Shrinkage modifies the algorithm's update of $f_{km}(x)$ instead with the scaled
addition $\nu \cdot \sum_{j=1}^{J_m} \gamma_{jkm} I(x \in R_{jkm})$, where the constant $\nu$ is between 0 and 1.
Smaller values of $\nu$ lead to greater rates of training errors, assuming that $M$ is fixed, but that in general
$\nu$ and $M$ are inversely related when the error is  fixed.
However, despite the greater rate of training error with small values of $\nu$, very small values ($\nu < 0.1$)
typically lead to better generalization and performance on test data.

%---- TODO: Keep making changes
\section{Use case: Classification with Airline data}

Download the Airline dataset from: {\url{https://github.com/h2oai/h2o/blob/master/smalldata/airlines/allyears2k_headers.zip}} and save the .csv file to your working directory. Before running the Airline demo, review how to load data with H2O.

\subsection{Loading data}

Loading a dataset in R for use with H2O is slightly different from the usual methodology, as we must convert our datasets into \texttt{H2OParsedData} objects. For this example, download the toy weather dataset from {\url{https://raw.githubusercontent.com/h2oai/h2o/master/smalldata/weather.csv}}.  Load the data to your current working directory in your R Console (do this for any future dataset downloads), and then run the following command.
\begin{lstlisting}[style=R]
weather.hex = h2o.uploadFile(h2o_server, path = "weather.csv", header = TRUE, sep = ",", key = "weather.hex")
\end{lstlisting}
\bigskip
\noindent
To see a brief summary of the data, run the following command.
\begin{lstlisting}[style=R]
summary(weather.hex)
\end{lstlisting}


\subsection{Performing a trial run}
Returning to the Airline dataset, load the dataset with H2O and select the variables to use to predict a chosen response. For example, model whether flights are delayed based on the departure's scheduled day of the week and day of the month.

\begin{lstlisting}[style=R]
# Load the data and prepare for modeling
air_train.hex = h2o.uploadFile(h2o_server, path = "AirlinesTrain.csv", header = TRUE, sep = ",", key = "airline_train.hex")

air_test.hex = h2o.uploadFile(h2o_server, path = "AirlinesTest.csv", header = TRUE, sep = ",", key = "airline_test.hex")

myX <- c("fDayofMonth", "fDayOfWeek")

\end{lstlisting}

Now, train the GBM model:

\begin{lstlisting}[style=R]
air.model <- h2o.gbm(y = "IsDepDelayed", x = myX, distribution="multinomial", data = air_train.hex, n.trees=100, interaction.depth=4, shrinkage=0.1, importance=TRUE)
\end{lstlisting}

\noindent
Since it is meant just as a trial run, the model contains only 100 trees. In this trial run, no validation set was specified, so by default, the model evaluates the entire training set.  To use n-fold validation, specify, for example, \texttt{nfolds=5}.

\subsection{Extracting and handling the results}

Now, extract the parameters of the model, examine the scoring process, and make predictions on the new data.

\begin{lstlisting}[style=R]
# View the specified parameters of your GBM model
air.model@model$params

# Examine the performance of the trained model
air.model

\end{lstlisting}
\noindent
The second command ({\texttt{air.model}}) returns the trained model's training and validation errors.
\\
\\
After generating a satisfactory model, use the \texttt{h2o.predict()} command to compute and store predictions on the new data, which can then be used for further tasks in the interactive modeling process.
\begin{lstlisting}[style=R]
# Perform classification on the held out data
prediction = h2o.predict(air.model, newdata=air_test.hex)

# Copy predictions from H2O to R
pred = as.data.frame(prediction)

head(pred)

\end{lstlisting}


\subsection{Web interface}

H2O R users have access to an intuitive web interface for H2O, Flow, to mirror the model building process in R. After loading data or training a model in R, point your browser to your IP address and port number (e.g., localhost:12345) to launch the web interface. From here, you can click on \textsc{Admin} $>$ \textsc{Jobs} to view specific details about your model. You can also click on \textsc{Data} $>$ \textsc{List All Frames} to view all current H2O frames.

\subsection{Variable importances}

To enable the variable importances option, use the additional argument \texttt{importance=TRUE}. This displays the absolute and relative predictive strength of each feature in the prediction task. From R, access these strengths using the command \texttt{air.model@model\$varimp}. You can also view a visualization of the variable
importances on the web interface.

\subsection{Supported Output}

The following algorithm outputs are supported:
\begin{itemize}
\item {\bf{Regression}}: Mean Squared Error (MSE), with an option to output variable importances or a Java POJO model
\item {\bf{Binary Classification}}: Confusion Matrix or Area Under Curve (AUC), with an option to output variable importances or a Java POJO model
\item {\bf{Classification}}: Confusion Matrix (with an option to output variable importances or a Java POJO model)
\end{itemize}

\subsection{Java model}

To access Java (POJO) code to use to build the current model in Java, click the \textsc{Preview POJO} button at the bottom of the model results. If the model is small enough, the code for the model displays within the GUI; larger models can be inspected after downloading the model.
\\
\\
To download the model:
\begin{enumerate}
\item Open the terminal window.
\item Create a directory where the model will be saved.
\item Set the new directory as the working directory.
\item Follow the curl and java compile commands displayed in the instructions at the top of the Java model.
\end{enumerate}

\subsection{Grid search for model comparison}

To support grid search capabilities for model tuning,  specify sets of values for parameter arguments to tweak certain parameters and observe changes in model behavior. The following is an example of a grid search:

\begin{lstlisting}[style=R]
air.grid <- h2o.gbm(y = "IsDepDelayed", x = myX, distribution="multinomial", data = air_train.hex, n.trees=c(5,10,15), interaction.depth=c(2,3,4), shrinkage=c(0.1,0.2))
\end{lstlisting}
\noindent
This example specifies three different tree numbers, three different tree sizes, and two different shrinkage values. This grid search model effectively trains eighteen different models over the possible combinations of these parameters. Of course, sets of other parameters can be specified for a larger space of models. This allows for more subtle insights in the model tuning and selection process, especially during inspection and comparison of the trained models after the grid search process is complete. To decide how and when to choose different parameter configurations in a grid search, refer to the beginning section for parameter descriptions and suggested values.

\begin{lstlisting}[style=R]
# print out all prediction errors and run times of the models
air.grid
air.grid@model

# print out a *short* summary of each of the models (indexed by parameter)
air.grid@sumtable

# print out *full* summary of each of the models
all_params = lapply(air.grid@model, function(x) { x@model$params })
all_params

# access a particular parameter across all models
shrinkages = lapply(air.grid@model, function(x) { x@model$params$shrinkage })

shrinkages
\end{lstlisting}

\section{Conclusion}

Gradient boosted machines sequentially fit new models to provide a more accurate estimate of a response variable in supervised learning tasks such as regression and classification. Though notorious for being difficult to distribute and parallelize, H2O's GBM offers both features in its framework, along with a straightforward environment for model tuning and selection.

\newpage

\section{References}

Click, Cliff and SriSatish Ambati. {\textbf{``Cliff Click Explains GBM at Netflix October 10 2013``}}\\
 \url{http://www.slideshare.net/0xdata/cliff-click-explains-gbm} SlideShare (2013).
\\
\\
Dietterich, Thomas G, and Eun Bae Kong. {\textbf{``Machine Learning Bias, Statistical Bias, and Statistical Variance of Decision Tree Algorithms.``}} \\
\url{http://www.iiia.csic.es/~vtorra/tr-bias.pdf} ML-95 255 (1995).
\\
\\
Elith, Jane, John R Leathwick, and Trevor Hastie. {\textbf{``A Working Guide to Boosted Regression Trees.``}} \\\url{http://onlinelibrary.wiley.com/doi/10.1111/j.1365-2656.2008.01390.x/abstract} Journal of Animal Ecology 77.4 (2008): 802-813
\\
\\
Friedman, Jerome H. {\textbf{``Greedy Function Approximation: A Gradient Boosting Machine.``}}\\
 \url{http://statweb.stanford.edu/~jhf/ftp/trebst.pdf} Annals of Statistics (2001): 1189-1232.
\\
\\
Friedman, Jerome, Trevor Hastie, Saharon Rosset, Robert Tibshirani, and Ji Zhu. {\textbf{``Discussion of Boosting Papers.``}}\\
\url{http://web.stanford.edu/~hastie/Papers/boost_discussion.pdf} Ann. Statist 32 (2004): 102-107
\\
\\
Friedman, Jerome, Trevor Hastie, and Robert Tibshirani. {\textbf{``Additive Logistic Regression: A Statistical View of Boosting (With Discussion and a Rejoinder by the Authors).``}}\\
 \url{http://projecteuclid.org/DPubS?service=UI&version=1.0&verb=Display&handle=euclid.aos/1016218223} The Annals of Statistics 28.2 (2000): 337-407
\\
\\
Hastie, Trevor, Robert Tibshirani, and J Jerome H Friedman. {\textbf{``The Elements of Statistical Learning``}}\\
  \url{http://statweb.stanford.edu/~tibs/ElemStatLearn/printings/ESLII_print10.pdf}. Vol.1. N.p., page 339: Springer New York, 2001.
\\
\\




\end{document}
