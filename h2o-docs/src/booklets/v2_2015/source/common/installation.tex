\section{Installation} 

The easiest way to directly install H2O is  via an R or Python package.

({\bf{Note}}: This document was created with H2O version \waterVersion.)

\subsection{Installation in R}

To load a recent H2O package from CRAN, run:

\begin{lstlisting}[style=R]
install.packages("h2o")
\end{lstlisting}

({\bf{Note}}: The version of H2O in CRAN is often one release behind the current version.

Alternatively, you can (and should for this tutorial) download the
latest stable H2O-3 build from the H2O download page:

\begin{enumerate}
\item Go to {\url{http://h2o.ai/download}}.
\item Choose the latest stable H2O-3 build.
\item Click the "Install in R" tab.
\item Copy and paste the commands into your R session.
\end{enumerate}

\bigskip
After H2O is installed on your system, verify the installation:

\begin{lstlisting}[style=R]
library(h2o)

# Start H2O on your local machine using all available cores.
# By default, CRAN policies limit use to only 2 cores.
h2o.init(nthreads = -1)

# Get help
?h2o.glm

# Show a demo
demo(h2o.glm)
\end{lstlisting}

\subsection{Installation in Python}

To load a recent H2O package from PyPI, run:

\begin{lstlisting}[style=python]
pip install h2o
\end{lstlisting}

Alternatively, you can (and should for this tutorial) download the
latest stable H2O-3 build from the H2O download page:

\begin{enumerate}
\item Go to {\url{http://h2o.ai/download}}.
\item Choose the latest stable H2O-3 build.
\item Click the "Install in Python" tab.
\item Copy and paste the commands into your Python session.
\end{enumerate}

\bigskip
After H2O is installed, verify the installation:

\begin{lstlisting}[style=python]
import h2o

# Start H2O on your local machine
h2o.init()

# Get help
help(h2o.glm)

# Show a demo
h2o.demo("glm")
\end{lstlisting}

\subsection{Pointing to a different H2O cluster}

Following the instructions in the previous sections create a one-node H2O cluster on your local machine. 

To connect to an established H2O cluster (in a multi-node Hadoop environment, for example) specify the IP address and port number for the established cluster using the \texttt{ip} and \texttt{port} parameters in the \texttt{h2o.init()} command: 

\begin{lstlisting}[style=R]
h2o.init(ip=123.45.67.89, port=54321)
\end{lstlisting}
