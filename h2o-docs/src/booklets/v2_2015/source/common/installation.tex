\section{Installation} 

The easiest way to install H2O is directly via an R or Python package.
(Note this document was created with H2O version \waterVersion.)

\subsection{Installation in R}

You can load a recent H2O package from CRAN by running (be advised
that the version of H2O in CRAN often lags by a release):

\begin{lstlisting}[style=R]
install.packages("h2o")
\end{lstlisting}

Alternatively, you can (and should for this tutorial) download the
latest stable H2O-3 build from the H2O download page as directed
below:

\begin{itemize}
\item Go to http://h2o.ai/download
\item Choose the latest stable H2O-3 build
\item Click on the "Install in R" tab
\item Copy and paste the instructions into your R session
\end{itemize}

\bigskip
Once you have H2O installed on your system, verify the installation
with the following instructions:

\begin{lstlisting}[style=R]
library(h2o)

# Start H2O on your local machine using all available cores.
# By default, CRAN policies limit use to only 2 cores.
h2o.init(nthreads = -1)

# Get help
?h2o.glm

# Show a demo
demo(h2o.glm)
\end{lstlisting}

\subsection{Installation in Python}

You can load a recent H2O package from PyPI by running:

\begin{lstlisting}[style=python]
pip install h2o
\end{lstlisting}

Alternatively, you can (and should for this tutorial) download the
latest stable H2O-3 build from the H2O download page as directed
below:

\begin{itemize}
\item Go to http://h2o.ai/download
\item Choose the latest stable H2O-3 build
\item Click on the "Install in Python" tab
\item Copy and paste the instructions into your python session
\end{itemize}

\bigskip
Once you have H2O installed on your system, verify the installation
with the following instructions:

\begin{lstlisting}[style=python]
import h2o

# Start H2O on your local machine
h2o.init()

# Get help
help(h2o.glm)

# Show a demo
h2o.demo("glm")
\end{lstlisting}

\subsection{Pointing to a different H2O cluster}

The above examples illustrate creating an H2O cluster of 1 node on
your local machine.

You can also set the \texttt{ip} and \texttt{port} parameters to tell
h2o.init() to connect to a pre-created H2O cluster (in a multi-node
Hadoop environment, for example).
