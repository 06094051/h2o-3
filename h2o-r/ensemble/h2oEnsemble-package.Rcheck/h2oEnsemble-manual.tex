\nonstopmode{}
\documentclass[a4paper]{book}
\usepackage[times,inconsolata,hyper]{Rd}
\usepackage{makeidx}
\usepackage[utf8,latin1]{inputenc}
% \usepackage{graphicx} % @USE GRAPHICX@
\makeindex{}
\begin{document}
\chapter*{}
\begin{center}
{\textbf{\huge Package `h2oEnsemble'}}
\par\bigskip{\large \today}
\end{center}
\begin{description}
\raggedright{}
\item[Type]\AsIs{Package}
\item[Title]\AsIs{H2O Ensemble Learning}
\item[Version]\AsIs{0.0.4}
\item[Date]\AsIs{2015-06-15}
\item[Author]\AsIs{Erin LeDell}
\item[Maintainer]\AsIs{Erin LeDell }\email{erin@h2o.ai}\AsIs{}
\item[Description]\AsIs{This package implements the Super Learner ensemble (stacking) algorithm using the H2O R interface to provide base learning algorithms.}
\item[License]\AsIs{Apache License (== 2.0)}
\item[Depends]\AsIs{R (>= 2.13.0), h2o (>= 3.1.0.99999)}
\item[NeedsCompilation]\AsIs{no}
\item[SystemRequirements]\AsIs{Java (>= 1.6)}
\item[Suggests]\AsIs{SuperLearner, cvAUC}
\item[URL]\AsIs{
}\url{https://github.com/h2oai/h2o-3/tree/master/h2o-r/ensemble/h2oEnsemble-package}\AsIs{}
\end{description}
\Rdcontents{\R{} topics documented:}
\inputencoding{utf8}
\HeaderA{h2oEnsemble-package}{H2O Ensemble Package}{h2oEnsemble.Rdash.package}
\aliasA{h2oEnsemble}{h2oEnsemble-package}{h2oEnsemble}
\keyword{models}{h2oEnsemble-package}
%
\begin{Description}\relax
This package implements the super learner ensemble (stacking) algorithm using the H2O R interface to provide base learning algorithms. 
\end{Description}
%
\begin{Details}\relax

\Tabular{ll}{
Package: & h2oEnsemble\\{}
Type: & Package\\{}
Version: & 0.0.4\\{}
Date: & 2015-06-15\\{}
License: & Apache License (== 2.0)\\{}
}
\end{Details}
%
\begin{Author}\relax
Erin LeDell

Maintainer: Erin LeDell <erin@h2o.ai>
\end{Author}
%
\begin{References}\relax
van der Laan, M. J., Polley, E. C. and Hubbard, A. E. (2007) Super Learner, Statistical Applications of Genetics and Molecular Biology, 6, article 25. \\{}
\url{http://dx.doi.org/10.2202/1544-6115.1309}\\{}
\url{http://biostats.bepress.com/ucbbiostat/paper222}\\{}
\\{}
Breiman, L. (1996) Stacked Regressions, Machine Learning, 24:49–64.\\{}
\url{http://dx.doi.org/10.1007/BF00117832}\\{}
\url{http://statistics.berkeley.edu/sites/default/files/tech-reports/367.pdf}
\end{References}
%
\begin{SeeAlso}\relax
\code{\LinkA{SuperLearner}{SuperLearner}}, \code{\LinkA{subsemble}{subsemble}}
\end{SeeAlso}
%
\begin{Examples}
\begin{ExampleCode}
See \code{\LinkA{h2o.ensemble}{h2o.ensemble}} for examples.
\end{ExampleCode}
\end{Examples}
\inputencoding{utf8}
\HeaderA{h2o.ensemble}{H2O Ensemble}{h2o.ensemble}
%
\begin{Description}\relax
This function creates a "Super Learner" ensemble using the H2O base learning algorithms specified by the user.
\end{Description}
%
\begin{Usage}
\begin{verbatim}
h2o.ensemble(x, y, training_frame, family = "binomial", 
  model_id = "", validation_frame = NULL,
  type = "superlearner",
  learner = c("h2o.glm.wrapper", "h2o.randomForest.wrapper", "h2o.gbm.wrapper", "h2o.deeplearning.wrapper"), 
  metalearner = "h2o.glm.wrapper", 
  cvControl = list(V = 5, shuffle = TRUE), 
  seed = 1, parallel = "seq")
\end{verbatim}
\end{Usage}
%
\begin{Arguments}
\begin{ldescription}
\item[\code{x}] 
A vector containing the names of the predictors in the model.

\item[\code{y}] 
The name of the response variable in the model.

\item[\code{training\_frame}] 
(Optional) An \code{\LinkA{H2OFrame}{H2OFrame.Rdash.class}} object containing the variables in the model.

\item[\code{family}] 
A description of the error distribution and link function to be used in the model.  This must be a character string.  Currently supports \code{"binomial"} and \code{"gaussian"}.  

\item[\code{model\_id}] 
(Optional) The unique id assigned to the resulting model. If none is given, an id will automatically be generated.

\item[\code{validation\_frame}] 
(Optional) An \code{\LinkA{H2OFrame}{H2OFrame.Rdash.class}} object indicating the validation dataset used to contruct the confusion matrix. If left blank, this defaults to the training data when \code{nfolds = 0}.

\item[\code{type}] 
A string naming the type of ensemble to be used.  The default is \code{"superlearner"} for the Super Learner (stacking) algorithm.  This is the only type supported at this time.

\item[\code{learner}] 
A string or character vector naming the prediction algorithm(s) used to train the base models for the ensemble.  The functions must have the same format as the h2o wrapper functions.

\item[\code{metalearner}] 
A string specifying the prediction algorithm used to learn the optimal combination of the base learners.  Supports both h2o and SuperLearner wrapper functions.

\item[\code{cvControl}] 
A list of parameters to control the cross-validation process. The \code{V} parameter is an integer representing the number of cross-validation folds and defaults to 10. Other parmeters are \code{stratifyCV} and \code{shuffle}, which are not yet enabled. 

\item[\code{seed}] 
A random seed to be set (integer); defaults to 1. If \code{NULL}, then a random seed will not be set.  The seed is set prior to creating the CV folds and prior to model training for base learning and metalearning.

\item[\code{parallel}] 
A character string specifying optional parallelization. Use \code{"seq"} for sequential computation (the default) of the cross-validation and base learning steps. Use \code{"multicore"} to perform the V-fold (internal) cross-validation step as well as the final base learning step in parallel over all available cores. Or parallel can be a snow cluster object. Both parallel options use the built-in functionality of the R core "parallel" package.  Currently, only \code{"seq"} is compatible with the parallelized H2O algorithms, so this argument may be removed or modified in the future.

\end{ldescription}
\end{Arguments}
%
\begin{Value}

\begin{ldescription}
\item[\code{x}] 
A vector containing the names of the predictors in the model.

\item[\code{y}] 
The name of the response variable in the model.

\item[\code{family}] 
Returns the \code{family} argument from above.  

\item[\code{cvControl}] 
Returns the \code{cvControl} argument from above.

\item[\code{folds}] 
A vector of fold ids for each observation, ordered by row index.  The number of unique fold ids is specified in \code{cvControl\$V}.   

\item[\code{ylim}] 
Returns range of \code{y}.

\item[\code{seed}] 
An integer. Returns \code{seed} argument from above.

\item[\code{parallel}] 
An character vector. Returns \code{character} argument from above.

\item[\code{basefits}] 
A list of H2O models, each of which are trained using the \code{data} object.  The length of this list is equal to the number of base learners in the \code{learner} argument.

\item[\code{metafit}] 
The predictive model which is learned by regressing \code{y} on \code{Z} (see description of \code{Z} below).  The type of model is specified using the \code{metalearner} argument.

\item[\code{Z}] 
The Z matrix (the cross-validated predicted values for each base learner).  In the stacking ensemble literature, this is known as the "level-one" data and is the design matrix used to train the metalearner.

\item[\code{runtime}] 
A list of runtimes for various steps of the algorithm.  The list contains \code{cv}, \code{metalearning}, \code{baselearning} and \code{total} elements.  The \code{cv} element is the time it takes to create the \code{Z} matrix (see above).  The \code{metalearning} element is the training time for the metalearning step.  The \code{baselearning} element is a list of training times for each of the models in the ensemble.  The time to run the entire \code{h2o.ensemble} function is given in \code{total}.

\item[\code{h2o\_version}] 
The version of the h2o R package.

\item[\code{h2oEnsemble\_version}] 
The version of the h2oEnsemble R package.

\end{ldescription}
\end{Value}
%
\begin{Note}\relax
Using an h2o algorithm wrapper function as the metalearner is not yet producing good results.  For now, it is recommended to use the \code{\LinkA{SL.glm}{SL.glm}} function as the metalearner.
\end{Note}
%
\begin{Author}\relax
Erin LeDell \email{erin@h2o.ai}
\end{Author}
%
\begin{References}\relax
van der Laan, M. J., Polley, E. C. and Hubbard, A. E. (2007) Super Learner, Statistical Applications of Genetics and Molecular Biology, 6, article 25. \\{}
\url{http://dx.doi.org/10.2202/1544-6115.1309}\\{}
\url{http://biostats.bepress.com/ucbbiostat/paper222}\\{}
\\{}
Breiman, L. (1996) Stacked Regressions, Machine Learning, 24:49–64.\\{}
\url{http://dx.doi.org/10.1007/BF00117832}\\{}
\url{http://statistics.berkeley.edu/sites/default/files/tech-reports/367.pdf}
\end{References}
%
\begin{SeeAlso}\relax
\code{\LinkA{SuperLearner}{SuperLearner}}, \code{\LinkA{subsemble}{subsemble}}
\end{SeeAlso}
%
\begin{Examples}
\begin{ExampleCode}
## Not run: 
    
# An example of binary classification using h2o.ensemble

library(h2oEnsemble)  # requires version >=0.0.4 of h2oEnsemble
library(SuperLearner)  # For metalearner such as "SL.glm"
library(cvAUC)  # Used to calculate test set AUC (requires version >=1.0.1 of cvAUC)
localH2O <-  h2o.init(ip = "localhost", port = 54321, startH2O = TRUE, nthreads = -1)


# Import a sample binary outcome train/test set into R
train <- read.table("http://www.stat.berkeley.edu/~ledell/data/higgs_5k.csv", sep=",")
test <- read.table("http://www.stat.berkeley.edu/~ledell/data/higgs_test_5k.csv", sep=",")


# Convert R data.frames into H2O parsed data objects
training_frame <- as.h2o(localH2O, train)
validation_frame <- as.h2o(localH2O, test)
y <- "V1"
x <- setdiff(names(training_frame), y)
family <- "binomial"
training_frame[,c(y)] <- as.factor(training_frame[,c(y)])  #Force Binary classification
validation_frame[,c(y)] <- as.factor(validation_frame[,c(y)])  # check to validate that this guarantees the same 0/1 mapping?



# Create a custom base learner library & specify the metalearner
h2o.randomForest.1 <- function(..., ntrees = 1000, nbins = 100, seed = 1) h2o.randomForest.wrapper(..., ntrees = ntrees, nbins = nbins, seed = seed)
h2o.deeplearning.1 <- function(..., hidden = c(500,500), activation = "Rectifier", seed = 1)  h2o.deeplearning.wrapper(..., hidden = hidden, activation = activation, seed = seed)
h2o.deeplearning.2 <- function(..., hidden = c(200,200,200), activation = "Tanh", seed = 1)  h2o.deeplearning.wrapper(..., hidden = hidden, activation = activation, seed = seed)
learner <- c("h2o.randomForest.1", "h2o.deeplearning.1", "h2o.deeplearning.2")
metalearner <- "SL.glm"



# Train the ensemble using 4-fold CV to generate level-one data
# More CV folds will take longer to train, but should increase performance
fit <- h2o.ensemble(x = x, y = y, training_frame = training_frame, 
                    family = family, 
                    learner = learner, metalearner = metalearner,
                    cvControl = list(V=4))


# Generate predictions on the test set
pred <- predict.h2o.ensemble(fit, validation_frame)
labels <- as.data.frame(validation_frame[,c(y)])[,1]


# Ensemble test AUC 
AUC(predictions=as.data.frame(pred$pred)[,1], labels=labels)
# 0.7681649 (h2o-2)
# 0.7372054 (h2o-3)
# 0.7771959 (h2o-3)


# Base learner test AUC (for comparison)
L <- length(learner)
sapply(seq(L), function(l) AUC(predictions = as.data.frame(pred$basepred)[,l], labels = labels)) 
# 0.7583084 0.7145333 0.7123253 (h2o-2)
# 0.6957427 0.6578448 0.6428909 (h2o-3)
# 0.7740217 0.7191073 0.7156636 (h2o-3)

# Note that the ensemble results above are not reproducible since 
# h2o.deeplearning is not reproducible when using multiple cores.
# For reproducible results, use h2o.init(nthreads = 1)


## End(Not run)
\end{ExampleCode}
\end{Examples}
\inputencoding{utf8}
\HeaderA{h2o.example.wrapper}{Wrapper functions for h2o algorithms}{h2o.example.wrapper}
\aliasA{h2o.deeplearning.wrapper}{h2o.example.wrapper}{h2o.deeplearning.wrapper}
\aliasA{h2o.gbm.wrapper}{h2o.example.wrapper}{h2o.gbm.wrapper}
\aliasA{h2o.glm.wrapper}{h2o.example.wrapper}{h2o.glm.wrapper}
\aliasA{h2o.randomForest.wrapper}{h2o.example.wrapper}{h2o.randomForest.wrapper}
\keyword{utilities}{h2o.example.wrapper}
%
\begin{Description}\relax
This is an example h2o algorithm wrapper function.
\end{Description}
%
\begin{Usage}
\begin{verbatim}
h2o.example.wrapper(x, y, data, key = "", family = "binomial", ...)
\end{verbatim}
\end{Usage}
%
\begin{Arguments}
\begin{ldescription}
\item[\code{x}] 
A vector containing the names of the predictors in the model.

\item[\code{y}] 
The name of the response variable in the model.

\item[\code{data}] 
An \code{\LinkA{H2OParsedData}{H2OParsedData.Rdash.class}} object containing the variables in the model.

\item[\code{key}] 
(Optional) The unique hex key assigned to the resulting model. If none is given, a key will automatically be generated.


\item[\code{family}] 
A description of the error distribution and link function to be used in the model.  This must be a character string.  Currently supports \code{"binomial"} and \code{"gaussian"}.  


\end{ldescription}
\end{Arguments}
%
\begin{Value}
An H2O model.
\end{Value}
%
\begin{Author}\relax
Erin LeDell \email{ledell@berkeley.edu}
\end{Author}
%
\begin{SeeAlso}\relax
\code{\LinkA{h2o.ensemble}{h2o.ensemble}}
\end{SeeAlso}
\inputencoding{utf8}
\HeaderA{predict.h2o.ensemble}{Predict method for an 'h2o.ensemble' object.}{predict.h2o.ensemble}
%
\begin{Description}\relax
Obtains predictions on a new data set from a \code{\LinkA{h2o.ensemble}{h2o.ensemble}} fit.  
\end{Description}
%
\begin{Usage}
\begin{verbatim}
## S3 method for class 'h2o.ensemble'
predict(object, newdata)
\end{verbatim}
\end{Usage}
%
\begin{Arguments}
\begin{ldescription}
\item[\code{object}] 
An object of class 'h2o.ensemble', which is returned from the \code{\LinkA{h2o.ensemble}{h2o.ensemble}} function.

\item[\code{newdata}] 
A \code{\LinkA{H2OParsedData}{H2OParsedData.Rdash.class}} object in which to look for variables with which to predict. 


\end{ldescription}
\end{Arguments}
%
\begin{Value}
\begin{ldescription}
\item[\code{pred}] 
A vector of predicted values from ensemble fit.

\item[\code{basepred}] 
A \code{\LinkA{H2OParsedData}{H2OParsedData.Rdash.class}} object with the predicted values from each base learner algorithm for the rows in \code{newdata}.  

\end{ldescription}
\end{Value}
%
\begin{Author}\relax
Erin LeDell \email{ledell@berkeley.edu}
\end{Author}
%
\begin{SeeAlso}\relax
\code{\LinkA{h2o.ensemble}{h2o.ensemble}}
\end{SeeAlso}
%
\begin{Examples}
\begin{ExampleCode}
# See h2o.ensemble documentation for an example.

\end{ExampleCode}
\end{Examples}
\printindex{}
\end{document}
